%% Generated by Sphinx.
\def\sphinxdocclass{report}
\documentclass[letterpaper,10pt,english]{sphinxmanual}
\ifdefined\pdfpxdimen
   \let\sphinxpxdimen\pdfpxdimen\else\newdimen\sphinxpxdimen
\fi \sphinxpxdimen=.75bp\relax
\ifdefined\pdfimageresolution
    \pdfimageresolution= \numexpr \dimexpr1in\relax/\sphinxpxdimen\relax
\fi
%% let collapsible pdf bookmarks panel have high depth per default
\PassOptionsToPackage{bookmarksdepth=5}{hyperref}

\PassOptionsToPackage{booktabs}{sphinx}
\PassOptionsToPackage{colorrows}{sphinx}

\PassOptionsToPackage{warn}{textcomp}
\usepackage[utf8]{inputenc}
\ifdefined\DeclareUnicodeCharacter
% support both utf8 and utf8x syntaxes
  \ifdefined\DeclareUnicodeCharacterAsOptional
    \def\sphinxDUC#1{\DeclareUnicodeCharacter{"#1}}
  \else
    \let\sphinxDUC\DeclareUnicodeCharacter
  \fi
  \sphinxDUC{00A0}{\nobreakspace}
  \sphinxDUC{2500}{\sphinxunichar{2500}}
  \sphinxDUC{2502}{\sphinxunichar{2502}}
  \sphinxDUC{2514}{\sphinxunichar{2514}}
  \sphinxDUC{251C}{\sphinxunichar{251C}}
  \sphinxDUC{2572}{\textbackslash}
\fi
\usepackage{cmap}
\usepackage[T1]{fontenc}
\usepackage{amsmath,amssymb,amstext}
\usepackage{babel}



\usepackage{tgtermes}
\usepackage{tgheros}
\renewcommand{\ttdefault}{txtt}



\usepackage[Bjarne]{fncychap}
\usepackage{sphinx}

\fvset{fontsize=auto}
\usepackage{geometry}


% Include hyperref last.
\usepackage{hyperref}
% Fix anchor placement for figures with captions.
\usepackage{hypcap}% it must be loaded after hyperref.
% Set up styles of URL: it should be placed after hyperref.
\urlstyle{same}

\addto\captionsenglish{\renewcommand{\contentsname}{Contents:}}

\usepackage{sphinxmessages}
\setcounter{tocdepth}{1}



\title{MIT RailSim}
\date{Apr 17, 2024}
\release{}
\author{Mojtaba Yousefi, Jonah Stadtmauer}
\newcommand{\sphinxlogo}{\vbox{}}
\renewcommand{\releasename}{}
\makeindex
\begin{document}

\ifdefined\shorthandoff
  \ifnum\catcode`\=\string=\active\shorthandoff{=}\fi
  \ifnum\catcode`\"=\active\shorthandoff{"}\fi
\fi

\pagestyle{empty}
\sphinxmaketitle
\pagestyle{plain}
\sphinxtableofcontents
\pagestyle{normal}
\phantomsection\label{\detokenize{index::doc}}


\sphinxstepscope


\chapter{Setup Guide}
\label{\detokenize{setup_guide:setup-guide}}\label{\detokenize{setup_guide::doc}}
\sphinxAtStartPar
This section outlines the steps for setting up the project environment using Pipenv, a tool for managing package dependencies and virtual environments. It assumes that the user has a working installation of pip.


\section{Prerequisites}
\label{\detokenize{setup_guide:prerequisites}}\begin{itemize}
\item {} 
\sphinxAtStartPar
Ensure pip is installed by running \sphinxcode{\sphinxupquote{pip \sphinxhyphen{}\sphinxhyphen{}version}} in your terminal. If pip is not installed, follow the instructions on the \sphinxhref{https://pip.pypa.io/en/stable/installation/}{pip installation guide}.

\end{itemize}


\section{Installing Pipenv}
\label{\detokenize{setup_guide:installing-pipenv}}
\sphinxAtStartPar
Pipenv can be installed via pip with the following command:

\begin{sphinxVerbatim}[commandchars=\\\{\}]
pip\PYG{+w}{ }install\PYG{+w}{ }\PYGZhy{}\PYGZhy{}user\PYG{+w}{ }pipenv
\end{sphinxVerbatim}

\sphinxAtStartPar
This command installs Pipenv for the current user. The \sphinxcode{\sphinxupquote{\sphinxhyphen{}\sphinxhyphen{}user}} flag ensures that Pipenv is installed in the user’s directory and does not require system\sphinxhyphen{}wide installation.


\section{Setting Up the Project Environment}
\label{\detokenize{setup_guide:setting-up-the-project-environment}}
\sphinxAtStartPar
Navigate to the project’s root directory, where the \sphinxcode{\sphinxupquote{Pipfile}} and \sphinxcode{\sphinxupquote{Pipfile.lock}} are located, and execute the following command to install dependencies and set up the virtual environment:

\begin{sphinxVerbatim}[commandchars=\\\{\}]
pipenv\PYG{+w}{ }install
\end{sphinxVerbatim}


\section{Activating the Virtual Environment}
\label{\detokenize{setup_guide:activating-the-virtual-environment}}
\sphinxAtStartPar
Activate the virtual environment created by Pipenv using:

\begin{sphinxVerbatim}[commandchars=\\\{\}]
pipenv\PYG{+w}{ }shell
\end{sphinxVerbatim}


\section{Troubleshooting Python Version Errors}
\label{\detokenize{setup_guide:troubleshooting-python-version-errors}}
\sphinxAtStartPar
If you encounter an error about an unavailable Python version, such as:

\begin{sphinxVerbatim}[commandchars=\\\{\}]
Error:\PYG{+w}{ }the\PYG{+w}{ }specified\PYG{+w}{ }Python\PYG{+w}{ }version\PYG{+w}{ }\PYG{o}{(}\PYG{l+m}{3}.8\PYG{o}{)}\PYG{+w}{ }is\PYG{+w}{ }not\PYG{+w}{ }available\PYG{+w}{ }on\PYG{+w}{ }your\PYG{+w}{ }system.
\end{sphinxVerbatim}

\sphinxAtStartPar
It is recommended to use \sphinxcode{\sphinxupquote{pyenv}} to manage multiple Python versions. Installation instructions for \sphinxcode{\sphinxupquote{pyenv}} can be found at the \sphinxhref{https://github.com/pyenv/pyenv\#installation}{pyenv GitHub repository}.

\sphinxAtStartPar
After installing \sphinxcode{\sphinxupquote{pyenv}}, you may need to restart your shell or terminal to ensure \sphinxcode{\sphinxupquote{pyenv}} and \sphinxcode{\sphinxupquote{pipenv}} are correctly added to your PATH.

\sphinxAtStartPar
When \sphinxcode{\sphinxupquote{pyenv}} is installed, running \sphinxcode{\sphinxupquote{pipenv install}} again will prompt Pipenv to use \sphinxcode{\sphinxupquote{pyenv}} to install the missing Python version. Confirming this allows Pipenv to manage the required Python version for the project automatically.


\section{Conclusion}
\label{\detokenize{setup_guide:conclusion}}
\sphinxAtStartPar
This guide provided instructions for setting up your project environment with Pipenv, including installing dependencies and resolving Python version issues with \sphinxcode{\sphinxupquote{pyenv}}. For additional information, consult the Pipenv documentation or community forums.


\chapter{Indices and tables}
\label{\detokenize{index:indices-and-tables}}\begin{itemize}
\item {} 
\sphinxAtStartPar
\DUrole{xref,std,std-ref}{genindex}

\item {} 
\sphinxAtStartPar
\DUrole{xref,std,std-ref}{modindex}

\item {} 
\sphinxAtStartPar
\DUrole{xref,std,std-ref}{search}

\end{itemize}



\renewcommand{\indexname}{Index}
\printindex
\end{document}